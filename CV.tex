\documentclass[line, margin]{res}
\usepackage[T1]{fontenc}
\usepackage{csquotes}
\usepackage[colorlinks = true, urlcolor = blue]{hyperref}
\urlstyle{rm}
\topmargin=-.5in
\textheight = 9.5in

\begin{document}
 \moveleft.5\hoffset\centerline{\LARGE\bf \textsc {Curriculum Vitae}}
 
\begin{resume}
\section{\textsc{Contact Info}}
\textbf{\Large{Xinyu (Cicero) Lu}} \hfill Email: \href{mailto:cicerolu@jhu.edu}{cicerolu@jhu.edu}\\
	 Bloomberg 115, \hfill Url: \href{www.ciceroxlu.org}{www.ciceroxlu.org} \\
	 3400 N. Charles Street, \\
	 Baltimore, MD, 21218 
	 
\section{\textsc{Research Interests}}
\textbf{Exoplanets} modeling and connecting theoretical formation models with observational results from current and next generation telescopes are my primary interests. I also incorporate \textbf{machine learning algorithms} in astronomical data analyses. 

\section{\textsc{Education}} 
\textbf{Pursuing Ph.D.} in Physics and Astronomy, \emph{Johns Hopkins University} \hfill 2017-present\\
\textbf{Master of Arts in Physics and Astronomy}, Johns Hopkins University \hfill  05/2019\\
\textbf{B.S. in Physics}, University of California, Los Angeles \hfill  06/2017\\


\section{\textsc{Publication}}

\textbf{Lu, C. X.} and Naoz, S. ~\enquote{ Supernovae Kicks in Hierarchical Triple Systems}, 2019, \textsl{Monthly Notices of the Royal Astronomical Society},  484, 2, 1506--1525

Kilpatrick, Charles D.; Foley, Ryan J.; Abramson, Louis E.; Pan, Yen-Chen; \textbf{Lu, Cicero-Xinyu}; Williams, Peter; Treu, Tommaso; Siebert, Matthew R.; Fassnacht, Christopher D.; Max, Claire E.  \enquote{ On the Progenitor of the Type IIb Supernova 2016gkg}, 2017, \textsl{Monthly Notices of the Royal Astronomical Society}, 65, 4, 4650--4657

\section{\textsc{Presentation}}
Contributed Talk, ``\textit{Planet Occurrence as a Function of Metallicity to Probe Planet Formation}'', Chesapeake Bay Area Exoplanet Meeting, May 10th, 2019

Poster Presentation, ``Late-type star that host small planets are metal-rich'', The 21st Century H-R Diagram: The Power of Precision Photometry, April 23--26, 2018
%\begin{resume}
\section{\textsc{Scholarships} }
\textbf{UCLA Undergraduate Research Scholars Program Scholarship}: Awarded Scholarship by Van Tree Foundation, Sept. 2016 - July 2017\\
\textbf{UCLA Honors 2015 Summer Research Scholarship Recipient}: Awarded Stone Scholarship fund for summer research, Jun. 2015 - Sept. 2015

\section{\textsc{Teaching}}
2019 Spring: \textsc{Teaching Assistant}, AS.171.416/AS.171.610 Numerical Methods for Physicists, Lecturer: Prof.\ Kevin Schlaufman

2018 Fall \textsc{Teaching Assistant}, 	AS.020.334.01 Planets, Life and the Universe. 
Lecturer: Prof.\ Colin Norman \& Jocelyne DiRuggiero

2018 Spring: \textsc{Teaching Assistant}, AS.171.104 General Physics/Biology Majors II
Lecturer: Prof.\ Peter Armitage

2018 Spring: \textsc{Teacher}, \emph{AS.173.112 General Physics Laboratory I}

2017 Fall \textsc{Teaching Assistant}, \emph{AS.171.107 General Physics for Physical Sciences Majors (AL)} Lecturer: Prof.\ Robert Leheny \& Prof.\ Rosemary Wyse

2017 Fall \textsc{Teacher}, \emph{AS.173.111 General Physics Laboratory I}


\section{\textsc{Competition Awards}}
\textbf{Mathematical Contest in Modeling} \hfill  Global Contest\\
Meritorious Winner of 2015 (Top 10\% globally) \hfill 03/2015\\
$\bullet$ Implemented Monte Carlo algorithm and greedy-like search algorithm; Modeling with Matlab and Mathematica with Bayes analysis

\textbf{Teradata Hackathon} \hfill  Los Angeles\\
Won LA regional 4th place \hfill 10/2015\\
$\bullet$ Used 5 Gigabytes of static data from Teradata database and dynamic data scraped from twitter, designed an algorithm to optimize living condition for household buyers with five customizable parameters, such as price, reputations, living cost, etc

\section{\textsc{Skills}}
Python, Mathematica, \LaTeX{} , Matlab, C++, HTML5/CSS, PHP


\end{resume}
\end{document}